%---------------------------------------------------------------------------
%
%                          Vorlage der Arbeitsgruppe
%             Computer Vision and Pattern Recognition Group (CVPR)
%                           der Universität Münster
%                         http://cvpr.uni-muenster.de
%
%---------------------------------------------------------------------------
% Geeignet für:
%  - Seminararbeiten
%  - Bachelorarbeiten
%  - Masterarbeiten
%---------------------------------------------------------------------------
% Autoren:
%  - Daniel Tenbrinck
%  - Fabian Gigengack
%  - Michael Schmeing
%  - Lucas Franek
%  - Andreas Nienkötter
%---------------------------------------------------------------------------
% Version:
%  - 1.0.3 (05.10.2016)
%	 - Ersetzung von veralteten Befehlen durch Aktuelle
%	 - Einige ausführlichere Beispiele
%    - Einführung von listings
%    - Aktuelle Version der Eidesstattlichen Erklärung
%  - 1.0.2 (09.09.2011)
%    - Titelblatt um Matrikelnummer und Studiengang ergänzt
%  - 1.0.1 (05.07.2011)
%---------------------------------------------------------------------------
% 
% "THE BEER-WARE LICENSE" (Revision 42):
% The above mentioned authors wrote this file. As long as you retain this
% notice you can do whatever you want with this stuff. If we meet some day,
% and you think this stuff is worth it, you can buy us a beer in return.
%  --------------------------------------------------------------------------

\documentclass[a4paper, oneside,openany, 12pt, ngerman, listof=nochaptergap, bibliography=totoc,
,listof=totoc,listof=entryprefix,]{scrbook} % Layout-Einstellungen für das Dokument
%openany -> keine leeren Seiten nach chapter

\usepackage[utf8]{inputenc} % UTF-8 Codierung
\usepackage[ngerman]{babel} % Deutsche Beschriftung

\usepackage{graphicx} % Um Bilder einzufügen
%\usepackage{subfigure} % Um mehrere Bilder in eine figure einzufügen

\usepackage{verbatim} % Um Quellcode in das Dokument einzufügen.
\usepackage{xcolor} % Für Farben
\usepackage[linkbordercolor=blue]{hyperref} % Für Links im Dokument
\usepackage{algorithmic} % Für Pseudo-Code
\usepackage{algorithm} % Wrapper für Pseudo-Code
\usepackage[skip=2pt,font={small}, labelfont=bf]{caption} % kleine Bildunterschriften
%\setlength{\abovecaptionskip}{1pc}  % 1pc=12pt
%\setlength{\belowcaptionskip}{1pc}
\usepackage{subcaption}





\usepackage{geometry} % Für Feinanpassungen des Layouts

\usepackage{listings} % Für Code-Listings
\renewcommand{\lstlistingname}{Quelltext} %Ändert die Überschrift von Listing nach Quelltext

%Ergänzungen
\usepackage{todonotes}

\usepackage{comment}
\usepackage{amsthm}

% Einstellungen für Abstand an den Rändern
\geometry{a4paper,left=35mm,right=35mm,top=20mm,bottom=20mm, includeheadfoot}

%Abkürzungen
\usepackage{acronym}


%Verzeichnisse in Inhaltsverzeichnis
%\usepackage[nottoc]{tocbibind}

%Zähler abbildungen
\usepackage{chngcntr}
\counterwithout{figure}{chapter}

\usepackage{multicol}


\usepackage{enumitem} 
%\usepackage{abstract}

\renewcaptionname{ngerman}{\figurename}{Abb.}

%\BeforeStartingTOC[lof]{\def\autodot{:}}
%\BeforeStartingTOC[lot]{\def\autodot{:}}

\usepackage[strings]{underscore}
\newenvironment{abstract}{%
  \begin{center}\normalfont\usekomafont{disposition}Abstract\end{center}%
}{%
  \par
  \vfil\null% X
  \endtitlepage% X
}

\usepackage{tikz}%Automaten zeichnen
\usetikzlibrary{positioning,automata,trees,decorations.markings,arrows}

\begin{document}
%Einrückungen verhindern
\setlength{\parindent}{0em} 
\pagenumbering{roman}

% Titelblatt
\begin{titlepage}

\begin{centering}
\vspace{4cm}
%\vspace*{\fill}
\includegraphics[width=12cm]{./img/wwu-logo-neu.pdf}

\vspace{2cm} 

{\large
	Praktikum Copmuter Vision\\[0.5cm]
}

{\LARGE
	\textbf{Smart Camera Operator}\\[2cm]
}


{\large
	Vorgelegt von:\\[0.5cm]
}

{ \large
	Tobias Johanning, Markus Konetzny, Tabea Preusser \\[1cm]
}

{\large
	Münster, den 05.03.2020\\[1cm]
}



\end{centering}
\vfill


 


\end{titlepage}


% Inhaltsverzeichnis
\tableofcontents

\cleardoublepage


\newpage


\pagenumbering{arabic}
\mainmatter
% Die Hauptkapitel der Arbeit
\chapter{Einleitung}


\subsection*{Motivation}

Im Rahmen eines Projektes des Unternehmens Rimondo sollte die Abdeckung der Videoaufnahme von Reitsportturniere maximiert werden. Dazu sollen neben den internationalen auch von möglichst vielen, im besten Falle sogar allen, nationalen Turnieren in NRW Videoaufnahmen erstellt werden können, was später auch auf ganz Deutschland erweitert werden soll. Neben Bereitstellung der Turnieraufnahmen in der Mediathek von Rimondo sind auch Live-Aufnahmen in Zukunft geplant. Alleine 2018 fanden in NRW über 600 nationale und 82 internationale Reitsportturniere \footnote{Daten der Deutschen Reiterlichen Vereinigung E.V.} statt, von denen nur ein Bruchteil verfilmt wurde.
Dabei stellt das grundlegende Problem die hohen Personalkosten der Kameraleute dar, die für stundenlange professionelle Aufnahmen benötigt werden. Während dies für internationale Turniere, wie das \emph{Turnier der Sieger} in Münster, für Fernsehsender finanzierbar ist, ist dies für die kleineren, regionalen Turniere nicht rentabel. Da jedoch die Kundengruppe, welche aus den Turnierteilnehmern und deren Fans besteht, mit über 98.000 Mitgliedern in Reitsportvereinen in NRW als mögliche Kunden, großes Potenzial darstellt, wird eine Lösung benötigt. Aus diesem Grund wurden Projektgruppen von Informatikstudenten der WWU vor die Aufgabe gestellt die Kameraführung für Reitsportturniere zu automatisiert, indem ein \emph{Smart Camera Operator} im Rahmen eines Praktikums erstellt wird. Dies sollte eine Software umfassen, welche in der Lage ist den Kameramann für Turnieraufnahmen im Pferdesport zu ersetzen.



\subsection*{Aufgabenstellung und Zielsetzung}

Die Erstellung des \emph{Smart Camera Operators} zur Videoaufnahme von Reitsportturnieren sollte in drei Phasen unterteilt erreicht werden, wobei die Wahl der Methoden und Verfahren freigestellt wurde.

Die erste Phase hatte die Erstellung von Trainingsdaten von Pferden und Reitern zum Ziel, womit die Vorarbeit für die weitern Phasen geleistet wurde. Dazu hat Rimonod auf einer Online Platform ca. 1300 Frames aus Reitvideos zur Verfügung gestellt, auf denen mithilfe eines Labeling-Tools die Position der Reiter und Pferde eingetragen werden sollte. Jeder Teilnehmer des Praktikums hatte vom 24.10.2019 bis zum 14.11.2019 Zeit in den Bildern mindestens 200 Reiterpaare, genauer 400 Reiter oder Pferde, zu kennzeichnen. Diese Daten wurden anschließend von Rimondo ausgewertet und in Form einer Datenbank für die zweite Phase des Projektes weiterverwendet. 

Die zweite Phase, vom 19.11.2019 bis 19.12.2019, umfasste das Training eines Detektors anhand der zuvor erstellten Datenbank. Dieser soll in der Lage sein anhand von Reitern und Pferden auch Reiterpaare zu erkennen. Als Grundlage zum Training des Detektors wurde mehrstündiges Videomaterial von Weitwinkel-Kameras von Reitern aus derselben Reithalle zur Verfügung gestellt. Es sollen erste Videos erstellt werden, die anhand der Region of Interest (ROI) aus den Frames der detektierten Reiterpaare erstellt werden, wobei noch kein großer Wert auf flüssiges Tracking gelegt werden musste. Zudem soll dieser Detektor genutzt werden, um weitere Bilddaten schnell zu labeln und ihn mit diesen Daten weiter zu trainieren.

In der dritten Phase soll der endgültige \emph{Smart Camera Operator}, vom 20.12.2019 bis 6.3.2020 erstellt werden. Es wurde dazu weiteres Videomaterial zur Verfügung gestellt, welches mehrere Reithallen und einen Reitplatz abdeckte. Dabei soll die genauere Lokalisierung des Reiterpaars für eine robustere Implementierung des Trackers genutzt werden. Außerdem soll das Tracking der ROI flüssiger werden, wobei beispielsweise Filter sowie Entzerrung des Bildes eingesetzt werden können.

Unser persönliches Gruppenziel ist es einen robusten und vielseitigen Tracker mit Maschine Learning zu erstellen, der in der Lage ist einem einzelnen Reiterpaar zu folgen. Dabei liegt unser Fokus auf Problemen wie Verdeckung und Kreuzen von Reitern in unterschiedlichen Umgebungen, während wir uns weniger mit dem Perfomance Aspekt befassen wollen. Zudem wollen wir im Hinblick auf Benutzerfreundlichkeit das Labeling weiterer Daten und das Konvertieren der Videos mithilfe einer GUI umsetzen.




\subsection*{Aufbau der Arbeit}
In Kapitel 2 gehen wir auf die grundlegenden Entwurfsentscheidungen des Projektes ein, wobei die verwendete Hard- und Software sowie der Maschine Learning Ansatz vorgestellt werden. Weiter wird in Kapitel 3 der Anwendungsaufbau anhand der Funktionsweise des Detektors sowie der Benutzeroberfläche beleuchtet. Die Implementierung der einzelnen Phasen des Trackers wird anschließend in Kapitel 4 im Detail erläutert. Die erreichten Ziele und möglichen Verbesserungen mit dafür wichtigen Schwerpunkten werden in Kapitel 5 diskutiert und zum Schluss fasst Kapitel 6 das erfolgte Projekt zusammen.

\chapter{Entwurfsentscheidungen }
\label{ch:entwurf}

\section{Programmiersprache}
Für die Wahl der Programmiersprache haben wir zunächst die Vorkenntnisse aller Gruppenmitglieder in verschiedenen erlernten Sprachen abgeschätzt und betrachtet welche Sprachen einen einfachen Einstieg in das Thema Maschine Learning erlauben. Deshalb viel unsere Wahl trotz geringerer Geschwindigkeit im Vergleich zu C++ auf Python, da ohne viel Vorwissen schnell ein erster funktionierender Prototyp erstellt werden kann und wir zudem unserer Kenntnisse in dieser Sprache vertiefen können.

Für Python gibt es eine Vielzahl geeigneter Entwicklungsumgebungen wie Eclipse oder Visual Studio, die Wahl fiel jedoch auf die kostenlose Variante von PyCharm, mit der alle Gruppenmitglieder vertraut waren.

\section{Setup}
Mit dem Ansatz im Verlauf des Projektes Maschine Learning zu verwenden, wurde recht schnell deutlich wie wichtig eine gute Hardware Ausrüstung ist, um gute Performance beim Training des Models und bei der Detektion zu erreichen. Während die Umsetzung auch ausschließlich mit CPU möglich ist, besticht der Einsatz von GPU mit deutlicher Geschwindigkeit. Um diesen Vorteil zu nutzen, haben wir uns entschlossen für das Training mit dem kostenlosen Cloud-Service von Google Colaboratory in Verbindung mit Google Drive zu arbeiten, der ebenfalls kostenlos GPU Nutzung ermöglicht. Dabei stehen uns 25 GB Ram und je nach Zuweisung eine Tesla T4 GPU mit ca.8 GB oder eine Tesla K80 GPU mit ca. 12 GB zur Verfügung, was einen 25-fachen Geschwindigkeitsvorteil von GPU gegenüber CPU darstellt. Passend zur Wahl der Programmiersprache arbeitet Google Colab mit Jupyter Notebooks und hat bereits die meisten Bibliotheken installiert, wobei fehlende mit Kommandozeilen Befehlen noch hinzugefügt werden können. 

\section{Versionsverwaltung}
Um Änderungen an Dateien und Quellcode zu erfassen und sinnvoll zu strukturieren, bietet sich aufgrund von Zusammenarbeit mehrerer Gruppenmitglieder der Einsatz einer Versionsverwaltung an. Durch Organisation mit Zeitstempeln und Benutzerkennungen kann gemeinsam an Dateien gearbeitet, Änderungen nachvollzogen, Dateien wiederhergestellt und Zugriffe koordiniert werden. Die Wahl, mit welcher Versionsverwaltung das Projekt umgesetzt werden sollte, fiel auf Git als verteiltes System, welches wir in Form von Github nutzen. Dies hat den Grund, dass Git von der genutzten Entwicklungsumgebung PyCharm unterstützt wird und alle Mitglieder unserer Gruppe bereits Github durch vorherige Projekte vertraut waren, sodass relativ wenig Einarbeitungszeit erforderlich war. Die Einbindung in Google Colab ist mit Github ebenfalls möglich, indem das jeweilige Projekt geclont wird, was wir für die Datenbank und Maschine Learning benötigten.
\section{Toolselection}
	Opencv
	Numpy
	Scipy
	Scikit-learn
	tensorflow
	Pandas
	keras
	
	
\section{Maschine Learning}


Mask rcnn

Yolo
ImageAI



\chapter{Anwendungsaufbau}
\label{ch:anwendungsaufbau}
Im Folgenden wollen wir den Aufbau der implementierten Software bezüglich ihrer Funktionalität und der getroffenen Designentscheidungen der autonomen Kameraführung betrachten. 
\section{Programmaufbau}
\begin{figure}[h]
\includegraphics[width=\textwidth]{./img/Klassendiagramm.png}
\caption{Klassendiagramm vom Smart Camera Operator}
\label{fig:Klassendiagramm}
\end{figure}
Wir haben Frontend und Backend separiert, wobei die Klasse des Operators die Schnittstelle bildet. Für die Umsetzung der GUI wurde das Prinzip des Model View Controllers verwendet, um die Nutzereingaben effizient verarbeiten zu können. Der Smart Camera Operator besteht aus den beiden Hauptkomponenten des Detektors, der Training und Object Detection übernimmt, sowie des Videoverarbeitung, welche das Tracking eines Reiterpaares und die Videoqualität umfasst.

\newpage

\section{Video-Pipeline}

\begin{figure}[H]
\centering
\resizebox{.9\linewidth}{!}{
\begin{tikzpicture}[>=stealth,shorten >=1pt,auto,node distance=1cm,scale=1, transform shape,align=center,minimum size=7em,text width=6em,line width=0.14 em,rounded corners]


 \node[draw,fill=lightgray,fill opacity=0.3,text opacity=1] (A) at (0,0) 
 {Eingabevideo};
 \node[draw,right =of A,fill=lightgray,fill opacity=0.3,text opacity=1] (B) 
 {Extraktion der Frames};
 \node[draw,right =of B,fill=lightgray,fill opacity=0.3,text opacity=1] (C)
 {Detektion der Bounding Boxen von Reitern und Pferden};
 \node[draw,right =of C,fill=lightgray,fill opacity=0.3,text opacity=1] (D) 
 {Berechnung der Gesamtbox aller Reiterpaare};
 \node[draw, below =of D,fill=lightgray,fill opacity=0.3,text opacity=1] (E) 
 {Ratio der Box anpassen};
 \node[draw, below =of C,fill=lightgray,fill opacity=0.3,text opacity=1] (F) 
 {Sonderfälle am Bildrand beachten} ;
 \node[draw, below =of B,fill=lightgray,fill opacity=0.3,text opacity=1] (G) 
 {Zuschneiden der Frames} ;
 \node[draw, below =of A,fill=lightgray,fill opacity=0.3,text opacity=1] (H) 
 {Ausgabevideo} ;

 \path [->] (A) edge node[left] {} (B);
 \path [->](B) edge node[left] {} (C);
 \path [->](C) edge node[left] {} (D);
 \path [->](D) edge node[left] {} (E);
 \path [->](E) edge node[right] {} (F);
 \path [->](F) edge node[left] {} (G);
 \path [->](G) edge node[below] {} (H); 
\end{tikzpicture}
}
\vspace*{-1cm}
\caption{Video-Pipeline Phase 2}
\label{fig:VideoPipelinePhase2}
\end{figure}
\begin{figure}[H]
\centering
\resizebox{.9\linewidth}{!}{
\begin{tikzpicture}[>=stealth,shorten >=1pt,auto,node distance=1cm,scale=1, transform shape,align=center,,minimum size=7em,text width=6em,line width=0.14 em,rounded corners]

 \node[draw,fill=lightgray,fill opacity=0.3,text opacity=1] (A) at (0,0) 
 {Eingabevideo};
 \node[draw,right =of A,fill=lightgray,fill opacity=0.3,text opacity=1] (B) 
 {Extraktion der Frames};
 \node[draw,right =of B,fill=lightgray,fill opacity=0.3,text opacity=1] (C)
 {Verkleinern der Frames};
 \node[draw,right =of C,fill=lightgray,fill opacity=0.3,text opacity=1] (D)
 {Detektion der Bounding Boxen von Reitern und Pferden};
	%second row
	\node[draw,below =of D,fill=lightgray,fill opacity=0.3,text opacity=1] (E)
 {Vergrößern der detektierten Boxen}; 
 \node[draw,below =of C,fill=lightgray,fill opacity=0.3,text opacity=1] (F) 
 {Berechnung der Boxen von Reiterpaaren};
 \node[draw,below =of B,fill=lightgray,fill opacity=0.3,text opacity=1] (G)
 {Auswahl und Tracken eines Reiterpaares};
 \node[draw, below =of A,fill=lightgray,fill opacity=0.3,text opacity=1] (H) 
 {Ratio der Box anpassen};
 %third row
 \node[draw, below =of H,fill=lightgray,fill opacity=0.3,text opacity=1] (I) 
 {Sonderfälle am Bildrand beachten} ;
 \node[draw,below =of G,fill=lightgray,fill opacity=0.3,text opacity=1] (J)
 {Glätten aller Boxen};
 \node[draw, below =of F,fill=lightgray,fill opacity=0.3,text opacity=1] (K) 
 {Zuschneiden der Frames} ;
 \node[draw,below =of E,fill=lightgray,fill opacity=0.3,text opacity=1] (L)
 {Einfügen der Audiospur};
 \node[draw, below =of L,fill=lightgray,fill opacity=0.3,text opacity=1] (M) 
 {Ausgabevideo} ;

 \path [->] (A) edge node[left] {} (B);
 \path [->](B) edge node[left] {} (C);
 \path [->](C) edge node[left] {} (D);
 \path [->](D) edge node[left] {} (E);
 \path [->](E) edge node[right] {} (F);
 \path [->](F) edge node[left] {} (G);
 \path [->](G) edge node[below] {} (H); 
 \path [->](H) edge node[below] {} (I); 
 \path [->](I) edge node[below] {} (J); 
 \path [->](J) edge node[below] {} (K); 
 \path [->](K) edge node[below] {} (L); 
 \path [->](L) edge node[below] {} (M); 
\end{tikzpicture}
}
\caption{Video-Pipeline Phase 3}
\label{fig:VideoPipelinePhase3}
\end{figure}

Anhand der Ziele von den Phasen zwei und drei haben wir die Vorgehensweise mit einer Video-Pipeline geplant, welche wir dann schrittweise mit sinnvoller Aufgabenteilung umsetzen konnten. Die erste Version (Abb. \ref{fig:VideoPipelinePhase2}) umfasst dabei lediglich die grundlegenden Funktionen, die für das sinnvolle Bestimmen der ROI nötig sind. So wird hier pro Frame eines Videos für alle detektierten Reiterpaare ein gemeinsamer Bildausschnitt berechnet und diese ROI zu einem Ausgabevideo zusammengefügt.
Diese Video-Pipeline konnten wir dann in der dritten Phase (Abb. \ref{fig:VideoPipelinePhase3}) um weiter Aspekte erweitern, wobei diesmal der Fokus auf der Qualität des Ausgabevideos lag. Die Detektion konnte auf Frames mit geringerer Auflösung schneller durchgeführt werden, weshalb die Größe der gefundenen Boxen anschließen angepasst werden musste. Anstatt dem vorherigen Ansatz alle Reiterpaare zu verfolgen, wird nun ein einzelnes davon ausgewählt. Die Qualität des Ausgabevideos wird durch weniger sprunghafte Unterschiede in der Größe der aufeinanderfolgen ROI verbessert, sodass der Bildfluss flüssiger wird. Zusätzlich wird nun auch die Audiodatei des Ursprungsvideos dem Ausgabevideo hinzugefügt.


\section{Zustandsautomat}

Folgender Moore-Automat (Abb. \ref{fig:MoorAutomat}) soll zeigen, wie unser Smart Camera Operator idealerweise funktionieren soll.
Der Automat startet im \emph{missing} Zustand.
Die Transitionen \emph{gefunden} und \emph{nicht gefunden} beziehen sich auf ein Reiterpaar, das im Bild gefunden wurde - oder nicht.
In der unteren Hälfte jedes Zustands befinden sich die Aktionen, die durchgeführt werden solange sich der Automat in dem Zustand befindet.

\begin{figure}[h]
\centering
\resizebox{.9\linewidth}{!}{
\begin{tikzpicture}[->,>=stealth',shorten >=1pt,auto,node distance=6cm,thick,
	every node/.style={font=\sffamily\bfseries},
	main node/.style={rounded corners,fill=lightgray!20,draw,font=\sffamily}]

	\node[main node, rectangle split, rectangle split parts=2] (A) at (0,0)
	{\textbf{missing} \nodepart{second}zoom out, detect };
	\node[main node, rectangle split, rectangle split parts=2] (B) [right of =A]
	{\textbf{found} \nodepart{second} zoom in, track};
	\node[main node, rectangle split, rectangle split parts=2] (C) [right of =B]
	{\textbf{lost} \nodepart{second} stop track, detect};
 
	\path [->] (A) edge node[above]	 	{gefunden} 			(B);
	\path [->] (B) edge node[above]	 	{nicht gefunden}		(C);
	\path [->] (C) edge[bend right] node[above] {gefunden} 			(B)
		 edge[bend left] node[below] 	{nicht gefunden}		(A);

\end{tikzpicture}
}
\caption{Zustandsautomat vom Smart Camera Operator}
\label{fig:MoorAutomat}
\end{figure}



\chapter{Implementierung}
\label{ch:implementierung}

\section{Phase 1}

\section{Phase 2}
\subsection{Training des Model}
Bevor wir mit dem Training des Detektors angefangen haben, mussten wir uns anhand einiger Tutorials die Verwendung von Mask Rcnn anlesen und an einigen Beispielen testen. 

Im ersten Schritt haben wir die bereitgestellten Daten aus der Datenbank von Phase 1 aufbereitet, um diese zum Training zu verwenden. Dazu haben wir die Datenbank aufgeteilt, sodass pro Frame eine csv Datei mit allen Labeln im Format \dq image,label,x,y,width,height\dq existierte. Da jeder Frame mehrfach gelabelt wurde, haben wir mithilfe von Schwellwerten die zusammengehörigen Dopplungen bestimmt und davon den durchschnittlichen Wert abgespeichert. Anschließend haben wir die hohe Auflösung der Frames verringert, um die Trainingszeit zu verringern. Zuletzt haben wir die Daten in einer passenden Ordnerstruktur von \dq accepted_images\dq und dem darin liegenden Ordner \dq annotations\dq in ein Github Projekt eingebunden, sodass das Training von Google Colab aus erfolgen kann.

Im zweiten Schritt haben wir die benötigten Klassen zum Training erstellt. Dazu wurde die Klasse RiderConfig als Unterklasse der Config Klasse von Mask Rcnn erstellt, in der die Parameter individuell angepasst wurden. Wichtig war die Anzahl der zu detektierenden Klassen, die neben Reiter und Pferd auch den Hintergrund umfasst. Zudem wurde die Leistung der verfügbaren GPU angepasst sowie die Trainings- und Validierungsschritte pro Epoche. 
Weiter wurde die Klasse RiderDataset als Unterklasse der Dataset Klasse von Mask Rcnn erstellt. In dieser wurden die nötigen Funktionen zum Laden des Datensatzes in Trainings und Test Modus, zum Laden der Masken und extrahieren der Boxen anhand von Eckpunkten aus den cvs Dateien überschrieben.

Damit konnten wir im dritten Schritt das Model trainieren, wozu wir aufbauend auf den MS COCO Gewichten mittels Transfer Lernen Zeit sparen konnten. Für das Training haben wir den Datensatz in 70\% Trainings-,15 \% Test- und 15\% Validierungsdaten unterteilt. Die erste Version unseres Detektors haben wir mit 75 Epochen mit 500 Schritten trainiert, bis der Verlust pro Epoche sehr gering wurde.
\todo{genauer}

\subsection{Extraktion der Rois}
\subsection{weiters Labeln}


\section{Phase 3}

\subsection{GUI}
\subsection{weiteres Training des Model}
\subsection{Reiterpaar}
\subsection{Sprünge/Verschwinden}
\subsection{Verdecken}


\chapter{Ergebnisauswertung}
\label{ch:ergebnis}


\section{Planung}
Die vorgegebenen Phasen aus der Aufgabenstellung dieses Projektes stellten sich als sehr hilfreich für ein sinnvolles schrittweises Vorgehen und dessen Umsetzung heraus, besonders im Hinblick auf fehlendes Vorwissen aller Gruppenmitglieder in der Thematik Maschine Learning. Während die erste Phase problemlos ablief, gestaltete sich der praktische Einstieg in der zweiten Phase dagegen relativ langwierig, bis wir einen ersten Prototypen unseres Detektors erstellen konnten. Als Folge dessen war es uns nicht vollständig möglich das Ziel eines Reiterpaardetektors bereits komplett fertigzustellen, da dieser an einigen Stellen noch Fehler aufwies, welche jedoch direkt zu Beginn der dritten Phase behoben wurden. Ebenfalls sollte in der zweiten Phase das Labeling weiterer Daten umgesetzt werden, wobei die entsprechend vereinfachte Benutzung dieses Zieles erst mithilfe der GUI in Phase drei hinzugefügt wurde. Diese zeitlichen Verschiebungen konnten jedoch mit effizienter Aufgabenteilung gut bewältigt werden und stellten kein Hindernis für die letzte Phase dar.
Durch die recht allgemeine Formulierung der Ziele in Phase drei, die einen Fokus auf robustes und flüssiges Tracking legten, haben wir die daraus abgeleiteten Gruppenziele in den Vordergrund gestellt. Obwohl einige Konzepte, die aus Effizienz- oder Performancegründen verworfen wurden, erprobt wurden, konnten die Projektziele größtenteils eingehalten werden. 

\section{Umsetzung und Zielerreichung}
Im Folgende betrachten wir die Umsetzung der einzelnen Ziele und die Einstufung von bestehenden Fehler sowie deren Wichtigkeit im Anwendungskontext.



\subsection*{ Detektion}
Der erste Schritt für die autonome Kameraführung war die Erstellung eines Detektors für Reiter und Pferde sowie später auch für Reiterpaare. Diesen konnten wir erfolgreich mithilfe von Maschine Learning umsetzten, indem wir die erstellten Trainingsdaten der ersten Phase sowie Transfer Learning als Grundlage genutzt haben und den Detektor später um weitere Datensätze erweitert haben. Mit Wahl von Mask R-CNN haben wir schon am Ende der zweiten Phase ausreichende Genauigkeit erreicht, dass wir problemlos zusätzliche Daten mit weiteren Videomaterial labeln konnten, obwohl sich die Umgebung und die Reiterpaare unterschieden. Die Schnelligkeit des Labelings konnten wir durch eine graphische Benutzeroberfläche gewährleisten, mit welcher ein Nutzer die möglichen Detektionen  beurteilen kann. Dies stellt für den Einsatz der Software mit ungeschulten Nutzern einen großen Vorteil dar, die mit wenigen Interaktionen ein Video nach der Aufnahme konvertieren können. Trotz der erfolgten Erweiterung des Detektors ist es in jeder neuen Umgebung auch weiterhin nötig diesen weiter zu trainieren, da durchaus fehlerhafte Erkennungen auftreten können.
Das Detektieren von Reiterpaaren konnten wir simpel mithilfe der Überdeckung von Reiter und Pferd lösen. Als Folge dessen konnten wir einzelne Reiter und Pferde vom Tracking ausschließen, was in neuen Umgebungen robusteres Detektieren zur Folge hat.

Während unser Ziel eines robusten Detektors mit Mask R-CNN gut umsetzbar war, stellt die Performance trotz Nutzung von GPU einen Nachteil gegenüber anderen Frameworks wie YOLO dar. Im Hinblick auf Liveübertragungen müsste dieser Interessenskonflikt genauer beurteilt werden und Möglichkeit gefunden werden, dass nicht für jeden Frame Detektionen benötigt werden. Die getesteten Verbesserungen mithilfe von polynominaler Regression ergab zwar nicht die gesuchte Geschwindigkeit, das Verfahren konnte jedoch eine gute Videoqualität erzielen, weshalb sich ein weitere Überprüfung sicherlich lohnen würde.


\subsection*{ Tracking}
Um eine flüssige und robuste Verfolgung des einzelnen Reiterpaares stellte sich als größere Herausforderung heraus als wir zu Anfang angenommen hatten. Um dies dennoch zu ermöglichen haben wir viele grundlegende Problematiken betrachtet und diese versucht schrittweise zu lösen. 
Neben der Behandlung der Bildränder haben wir auch das Verdecken und das fehlerhafte oder fehlende Detektieren einzelnen Paaren behandelt. Nachdem wir Sprünge und Verschwinden der errechneten ROI bestimmt und einen Gaußfilter angewendet haben, konnten wir die Videoqualität zufriedenstellend verbessern. An dieser Stelle würde es sich anbieten andere Filter auszutesten um herauszufinden welcher das beste Verhältnis zwischen Rechenaufwand und Qualität leistet, der verwendete Gaußfilter hat für die Nachbearbeitung von Videos jedoch bereits eine gute Qualität geliefert.

Der erstellte Tracker weist an einigen Stellen noch geringe Fehler beim Verfolgen eines einzelnen Reiters auf, wenn die Verdeckung zu lange anhält, sodass auch das Kreuzen von Reitern nicht immer die gewünschten Ergebnisse liefert. 
Wir mussten feststellen, dass einige Aspekte von manueller Kameraführung im Allgemeinen schwer umsetzbar sind, da beispielsweise situationsbedingt entschieden wird, welches Reiterpaar gefilmt werden soll. 
Im Hinblick auf den Einsatz bei Reitsportturnieren sind die Einzelprüfungen jedoch der wichtigste Bereich, welchen wir am meisten optimiert haben.


\section{Ausblick}
Durch die zeitlichen Vorgaben konnten nicht alle Ideen und Ansätze ausprobiert werden, jedoch sehen wir in einigen Aspekten großes Potenzial zur Verbesserung und Erweiterung.

Der erstellte Detektor für Reiterpaare sollte sowohl für weitere Umgebungen, besonders im Außenbereich mit anderen Reitern und Pferden, weiter trainiert werden, als auch für verschiedene Reitdisziplinen wie beispielsweise Springreiten oder Voltigieren, bei denen sich die Bewegungsmuster verändern. 
Um eine noch robustere Verfolgung eines Reiterpaares zu ermöglichen sind weitere Sonderfälle, wie das Kreuzen und längeres Verdecken von Reitern, genauer zu beachten. Ebenfalls sind Ausnahmesituationen wie Stürze von Interesse, da in diesem Fall Reiter und Pferd nicht mehr als Paar erkannt würden.
Ein ausgiebiger Praxistest würde uns an dieser Stelle Aufschluss über nötige Erweiterungen und Problematiken geben.

Während für das Praktikum eine feste Position und Blickrichtung der Kamera vorgegeben waren, müsste der bestehende Smart Camera Operator im Hinblick auf den Einsatz für Live Tracking an einigen Stellen modifiziert werden, da dies andere Herausforderungen als die nachträgliche Bearbeitung von Aufzeichnungen mit sich bringt. Besonders müsste die Performance dafür deutlich erhöht werden, damit die Erkennung für ca. 25 fps erreicht werden kann. Davon ausgehend könnten dann auch Kamerabewegungen ermöglicht werden, die dem fokussierten Reiterpaar mittels Translationen und Rotationen folgt.

Die Verbesserung der Kameraführung hat noch großen Spielraum offen, bis ein professioneller Kameramann ersetzt werden kann, da diese durch ihre Erfahrung im Vorteil sind. Für einen Ausgleich könnte herausgefunden werden, an welchen Stellen sich ein Zoom auf den Reiter oder das Pferd für die jeweilige Turnierart lohnt, da Springreiten andere Anforderungen als Dressurreiten hat.



\chapter{Zusammenfassung }
\label{ch:zusammenfassung}


Im Rahmen des Praktikums Computer Vision konnten wir erfolgreich einen ersten Prototypen unseres Smart Camera Operators erstellen, der autonom die Kameraführung von Videoaufnahmen übernimmt. Dabei konnten wir neben einigen Methoden der klassischen Computer Vision besonders mit Maschine Learning die Objekterkennung von Reitern und Pferden umsetzen und mit unterschiedlichen Verfahren beim Tracken experimentieren. Die erstellte Software wurde im Interesse der Kundengruppe benutzerfreundlich konzipiert, sodass diese, mithilfe einer guten Kamera, einen Ersatz für einen unerfahrenen Kameramann darstellt. Obwohl sich die Fähigkeiten von professionellen Kameraleuten enorm von unserem Prototypen abheben, konnten wir in der kurzen Zeit erhebliche Fortschritte erzielen und sind positiv eingestellt, dass sich mit einer Weiterentwicklung eine bereits deutlich professionellere, autonome Kameraführung realisieren lassen würde.

Für das geplante Einsatzgebiet für Reitsportturniere in NRW, sehen wir auch großes Potenzial, dass nach einem Praxistest ein weiträumiger Einsatz für regionale Turniere möglich und rentabel ist. 
Die bisherige Nutzung des Smart Camera Operator für Dressurreiten, kann in Zukunft mit wenig Aufwand für weitere Turnierarten wie Springreiten oder Pferderennen erweitert werden. Ebenfalls sehen wir großes Potenzial, dass diese Anwendung auch für Trainingsstunden der Reitsportler von Interesse ist, um die eigene Leistung ohne Kameramann reflektieren zu können.

\appendix
%\addcontentsline{toc}{chapter}{Appendix}
%\addtocontents{toc}{\protect\contentsline{chapter}{Appendix:}{}}
\chapter{Videos Ergebnisse}
\chapter{Videos Siegerehrung}




% Literaturverzeichnis
\bibliographystyle{unsrtdin}

\bibliography{Quellen}




% Eidesstattliche Erklärung
%\include{chapter/EidesstattlicheErklaerung}

\end{document}