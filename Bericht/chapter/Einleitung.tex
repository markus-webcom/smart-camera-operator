\chapter{Einleitung}


\subsection*{Motivation}

Im Rahmen eines Projektes des Unternehmens Rimondo sollte ermöglicht werden, dass die Abdeckung der Videoaufnahme von Reitsportturniere maximiert werden soll. Dazu sollen neben den internationalen auch von möglichst vielen, im besten Falle sogar allen, nationalen Turnieren in NRW Videoaufnahmen erstellt werden können, was später auch auf ganz Deutschland erweitert werden soll. Neben Bereitstellung der Turnieraufnahmen in der Mediathek von Rimondo sind auch Live-Aufnahmen in der Zukunft geplant. Alleine 2018 fanden in NRW über 600 nationale und 82 internationale Reitsportturniere \footnote{Daten der Deutschen Reiterlichen Vereinigung E.V.} statt, von denen nur ein Bruchteil verfilmt wurde.
Dabei stellt das grundlegende Problem die hohen Personalkosten der Kameraleute dar, die für stundenlange professionelle Aufnahmen benötigt werden. Während dies für internationale Turniere, wie das „Turnier der Sieger“ in Münster, möglich über Fernsehsender zu finanzieren ist, sind die Kosten für die kleineren, regionalen Turniere nicht rentabel. Da jedoch die Kundengruppe, welche aus den Turnierteilnehmern und deren Fans besteht, mit über 98.000 Mitgliedern in Reitsportvereinen in NRW als mögliche Kunden, großes Potenzial darstellt, wird eine Lösung benötigt. Aus diesem Grund wurden Projektgruppen von Informatikstudenten der WWU vor die Aufgabe gestellt die Kameraführung für Reitsportturniere zu automatisiert, indem ein „Smart Camera Operator“ im Rahmen eines Praktikums erstellt wird. Dies sollte eine Software umfassen, welche in der Lage ist den Kameramann für Turnieraufnahmen im Pferdesport zu ersetzen.



\subsection*{Aufgabenstellung und Zielsetzung}

Die Erstellung des „Smart Camera Operators“ zur Videoaufnahme von Reitsportturnieren sollte in drei Phasen unterteilt erreicht werden, wobei die Wahl der Methoden und Verfahren freigestellt wurde.

Die erste Phase hatte die Erstellung von Trainingsdaten von Pferden und Reitern zum Ziel, womit die Vorarbeit für die weitern Phasen geleistet wurde. Dazu hat Rimonod auf einer Online Platform ca. 1300 Frames aus Reitvideos zur Verfügung gestellt, auf denen mithilfe eines Labeling-Tools die Position der Reiter und Pferde eingetragen werden sollte. Jeder Teilnehmer des Praktikums hatte vom 24.10.2019 bis zum 14.11.2019 Zeit in den Bildern mindestens 200 Reiterpaare, genauer 400 Reiter oder Pferde, zu kennzeichnen. Diese Daten wurden anschließend von Rimondo ausgewertet und in Form einer Datenbank für die zweite Phase des Projektes weiterverwendet. 

Die zweite Phase, vom 19.11.2019 bis 19.12.2019, umfasste das Training eines Detektors anhand der zuvor erstellten Datenbank. Dieser soll in der Lage sein anhand von Reitern und Pferden auch Reiterpaare zu erkennen. Als Grundlage zum Training des Detektors wurde mehrstündiges Videomaterial von Weitwinkel-Kameras von Reitern aus derselben Reithalle zum Testen zur Verfügung gestellt. Es sollen erste Videos erstellt werden, die anhand der Region of Interest (Roi) aus den Frames der detektierten Reiterpaare erstellt werden, wobei noch kein großer Wert auch flüssige Bildübergänge gelegt werden musste. Zudem soll dieser Detektor genutzt werden, um weitere Bilddaten zu labeln und den Detektor mit diesen Daten weiter zu trainieren.

Unser persönliches Ziel in dieser Phase ist einen bereits robusten Tracker mit dem bereitgestellten Videomaterial zu ermöglichen, weshalb wir direkt mit Maschine Learning beginnen wollen. Folglich lag der Fokus mehr auf der Detektion von Reiter und Pferd als konkret auf dem Reiterpferdepaar. Wir wollten direkt bei der Bestimmung der Rois besonders die Randfallbehandlung des Sichtfeldes einbeziehen und zunächst alle Reiterpaare gemeinsam im Bildausschnitt haben. Weiter wollten wir den Aspekt des Labelling weiterer Daten, für Videos und Bilder mithilfe einer GUI umsetzen.

In der dritten Phase soll der endgültige „Smart Camera Operator“, vom 20.12.2019 bis 6.3.2020 erstellt werden. Es wird dazu weites Videomaterial zur Verfügung gestellt, welches mehrere Reithallen Indoor und Outdoor abdeckt. Dabei soll die genauere Lokalisierung des Reiterpaars für eine robustere Implementierung des Trackers genutzt werden. Außerdem soll das Tracking der Rois flüssiger werden, wobei beispielsweise ein Kalmanfilter sowie Entzerrung des Bildes eingesetzt werden können.
Unser Ziel war es ein Flüssiges Video mithilfe eines passenden Filters zu erreichen, wobei wir einen Gaußfilter ausprobieren wollten. Weiter wollten wir den Reiterpaar Detektor verbessern, indem wir Sprünge und Verschwinden eines Paares einbeziehen, wodurch Probleme wie das Auftauchen im Spiegel gelöst werden sollen. Ebenfalls sollte das Verdeckungsproblem des beobachteten Paares in Angriff genommen werden, um ein einzelnes Reiterpaar erfolgreich zu tracken. Weiter sollte der Detektor durch weitere Daten aus Indoor und Outdoor Aufnahmen erweitert werden, um Vielseitigkeit zu erlangen



\subsection*{Aufbau der Arbeit}
In Kapitel 2 gehen wir auf die grundlegenden Entwurfsentscheidungen des Projektes ein, wobei die verwendete Hard- und Software sowie der Maschine Learning Ansatz vorgestellt. Weiter werden in Kapitel 3 der Anwendungsaufbau anhand der Funktionsweise des Detektors sowie der Benutzeroberfläche beleuchtet. Die Implementierung der einzelnen Phasen des Trackers wird in Kapitel 4 in Details erläutert und anschließend werden die erreichten Ziele in Kapitel 5 diskutiert. Zum Schluss fasst Kapitel 6 das erfolgte Projekt zusammen und gibt einen Ausblick auf mögliche Verbesserungen dafür wichtige Schwerpunkte.