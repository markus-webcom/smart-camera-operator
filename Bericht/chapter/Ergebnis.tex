\chapter{Ergebnisauswertung}
\label{ch:ergebnis}

Qualitative Auswertung:

•	Diskussion Zielerreichung

•	Untersuchung der Bedienung


•	Empfehlung noch nicht implementierter Funktionen

\vspace{1cm}

Bugs:

•	Prototyp

•	fehlender Praxistest

•	bisher nur Grundfunktionen

\vspace{1cm}

Verbesserungen:

•	Performance für Live Übertragung

•	Model Training erweitern

•	Festlegung welchem Reiterpaar situationsbedingt gefolgt werden soll

•	Bildentzerrung?


\vspace{1cm}
Zusammenfassung der Beobachtungen
	
	
•	Einstufung der Fehler und ihre Wichtigkeit

•	Fehlertoleranz

•	Ursachenanalyse
	
•	Verbesserungsvorschläge
	
•	Interssenskonflike(Performance und Zeitaufwand)
	
•	Beurteilung der Resourcen (Zeit, Vorwissen, erworbene Kenntnisse )

•	Effektivität

•	Effizienz

•	Zufriedenstellung/Akzeptanz

•	Aufgabenangemessenheit

•	Erlernbarkeit

•	Erwartungskonformität



\section{Planung}
Die vorgegebenen Phasen aus der Aufgabenstellung dieses Projektes stellten sich als sehr hilfreich für ein sinnvolles schrittweises Vorgehen und dessen Umsetzung heraus, besonders im Hinblick auf fehlendes Vorwissen aller Gruppenmitglieder in der Thematik Maschine Learning. Während die erste Phase problemlos ablief, gestaltete sich der praktische Einstieg in Phase zwei dagegen relativ langwierig bis wir einen ersten Prototypen unseres Detektors erstellen konnten. Als Folge dessen war es uns nicht vollständig möglich das Ziel eines Reiterpaardetektors bereits in Phase 2 fertigzustellen, da dieser an einigen Stellen noch Fehler aufwies, welche jedoch direkt zu Beginn der dritten Phase behoben wurden. Ebenfalls sollte in der zweiten Phase das Labelling weiterer Daten umgesetzt werden, wobei die entsprechend vereinfachte Benutzung dieses Zieles erst mithilfe der GUI in Phase drei hinzugefügt wurde. Diese zeitlichen Verschiebungen konnten jedoch in mit effizienter Aufgabenteilung gut bewältigt werden und stellten kein Hindernis für die letzte Phase dar.
Durch die recht allgemeine Formulierung der Ziele in Phase drei, die einen Fokus auf robustes und flüssiges Tracking, haben wir die daraus abgeleiteten Gruppenziele in den Vordergrund gestellt. Obwohl einige Konzepte, die aus Effizienz- oder Performancegründen verworfen wurden, erprobt wurden, konnten die Projektziele eingehalten werden. 

\section{Umsetzung und Zielerreichung}

\subsection*{Umsetzung GUI}
•	Genereller Überblick

•	Spezielle Bedienelemente

•	Ausführung vorgegebener Handlungsabläufe

•	Leichte Erlernbarkeit
\todo[inline]{Abschnitt fehlt}

\subsection*{Umsetzung Detektor}
\todo[inline]{Abschnitt fehlt}

\subsection*{Umsetzung Tracker}
\todo[inline]{Abschnitt fehlt}

\section{Verbesserungen}
\todo[inline]{Abschnitt fehlt}
Performance da Live Übertragung braucht mit ca. 25 fps schnellere Detektion

Tracken bei Verdeckung noch robuster

Tracken von Kreuzenden Reitern

erweitern des Models um mehr Datensätze für verschiedene Reitdisziplinen wie Springreiten

Kamerarotation/Translation erfordert andere Struktur da live Berechnungen nötig sind

Testen der Anwendung in der Praxis durchführen
