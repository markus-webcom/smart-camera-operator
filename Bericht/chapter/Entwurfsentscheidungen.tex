\chapter{Entwurfsentscheidungen }
\label{ch:entwurf}

\section{Programmiersprache}
Für die Wahl der Programmiersprache haben wir zunächst die Vorkenntnisse aller Gruppenmitglieder in verschiedenen erlernten Sprachen abgeschätzt und betrachtet welche Sprachen einen einfachen Einstieg in das Thema Maschine Learning erlauben. Deshalb viel unsere Wahl trotz geringerer Geschwindigkeit im Vergleich zu C++ auf Python, da ohne viel Vorwissen schnell ein erster funktionierender Prototyp erstellt werden kann und wir zudem unserer Kenntnisse in dieser Sprache vertiefen können.

Für Python gibt es eine Vielzahl geeigneter Entwicklungsumgebungen wie Eclipse oder Visual Studio, die Wahl fiel jedoch auf die kostenlose Variante von PyCharm, mit der alle Gruppenmitglieder vertraut waren.

\section{Setup}
Mit dem Ansatz im Verlauf des Projektes Maschine Learning zu verwenden, wurde recht schnell deutlich wie wichtig eine gute Hardware Ausrüstung ist, um gute Performance beim Training des Models und bei der Detektion zu erreichen. Während die Umsetzung auch ausschließlich mit CPU möglich ist, besticht der Einsatz von GPU mit deutlicher Geschwindigkeit. Um diesen Vorteil zu nutzen, haben wir uns entschlossen für das Training mit dem kostenlosen Cloud-Service von Google Colaboratory in Verbindung mit Google Drive zu arbeiten, der ebenfalls kostenlos GPU Nutzung ermöglicht. Dabei stehen uns 25 GB Ram und je nach Zuweisung eine Tesla T4 GPU mit ca.8 GB oder eine Tesla K80 GPU mit ca. 12 GB zur Verfügung, was einen 25-fachen Geschwindigkeitsvorteil von GPU gegenüber CPU darstellt. Passend zur Wahl der Programmiersprache arbeitet Google Colab mit Jupyter Notebooks und hat bereits die meisten Bibliotheken installiert, wobei fehlende mit Kommandozeilen Befehlen noch hinzugefügt werden können. 

\section{Versionsverwaltung}
Um Änderungen an Dateien und Quellcode zu erfassen und sinnvoll zu strukturieren, bietet sich aufgrund von Zusammenarbeit mehrerer Gruppenmitglieder der Einsatz einer Versionsverwaltung an. Durch Organisation mit Zeitstempeln und Benutzerkennungen kann gemeinsam an Dateien gearbeitet, Änderungen nachvollzogen, Dateien wiederhergestellt und Zugriffe koordiniert werden. Die Wahl, mit welcher Versionsverwaltung das Projekt umgesetzt werden sollte, fiel auf Git als verteiltes System, welches wir in Form von Github nutzen. Dies hat den Grund, dass Git von der genutzten Entwicklungsumgebung PyCharm unterstützt wird und alle Mitglieder unserer Gruppe bereits Github durch vorherige Projekte vertraut waren, sodass relativ wenig Einarbeitungszeit erforderlich war. Die Einbindung in Google Colab ist mit Github ebenfalls möglich, indem das jeweilige Projekt geclont wird, was wir für die Datenbank und Maschine Learning benötigten.
\section{Toolselection}
	Opencv
	Numpy
	Scipy
	Scikit-learn
	tensorflow
	Pandas
	keras
	
	
\section{Maschine Learning}
hier oder im nächsten Kapitel?
