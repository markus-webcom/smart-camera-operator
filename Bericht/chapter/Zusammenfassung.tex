\chapter{Zusammenfassung }
\label{ch:zusammenfassung}


Im Rahmen des Praktikums Computer Vision konnten wir erfolgreich einen ersten Prototypen unseres Smart Camera Operators erstellen, der autonom die Kameraführung von Videoaufnahmen übernimmt. Dabei konnten wir neben einigen Methoden der klassischen Computer Vision besonders mit Maschine Learning die Objekterkennung von Reitern und Pferden umsetzen und mit unterschiedlichen Verfahren beim Tracken experimentieren. Die erstellte Software wurde im Interesse der Kundengruppe benutzerfreundlich konzipiert, sodass diese, mithilfe einer guten Kamera, einen Ersatz für einen unerfahrenen Kameramann darstellt. Obwohl sich die Fähigkeiten von professionellen Kameraleuten enorm von unserem Prototypen abheben, konnten wir in der kurzen Zeit erhebliche Fortschritte erzielen und sind positiv eingestellt, dass sich mit einer Weiterentwicklung eine bereits deutlich professionellere, autonome Kameraführung realisieren lassen würde.

Für das geplante Einsatzgebiet für Reitsportturniere in NRW, sehen wir auch großes Potenzial, dass nach einem Praxistest ein weiträumiger Einsatz für regionale Turniere möglich und rentabel ist. 
Die bisherige Nutzung des Smart Camera Operator für Dressurreiten, kann in Zukunft mit wenig Aufwand für weitere Turnierarten wie Springreiten oder Pferderennen erweitert werden. Ebenfalls sehen wir großes Potenzial, dass diese Anwendung auch für Trainingsstunden der Reitsportler von Interesse ist, um die eigene Leistung ohne Kameramann reflektieren zu können.